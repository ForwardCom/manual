% chapter included in forwardcom.tex
\documentclass[forwardcom.tex]{subfiles}
\begin{document}
\RaggedRight


\section{Assembly language syntax} \label{assemblySyntax}
The definition of a new instruction set should include the definition of a standardized assembly language syntax. The syntax should be suitable for human processing, not only for machine processing. We must avoid a situation similar to the x86 environment where many different syntaxes are in use, with different instruction names and different orders of the operands. 
\vspace{2mm}

ForwardCom will use a function-style syntax in order to make it more intelligible to high-level language programmers and to avoid any confusion over what is source and destination. 

\vspace{2mm}
There is only one instruction on each line. A suffix is added to instruction names, separated by a dot, to indicate the operand type: 8, 16, 32, 64, 128 for integer operand size, and f, d, q for single, double and quadruple precision floating point operands. For example:
\begin{lstlisting}
   r0 = add.32 (r1, r2) // r0 = r1 + r2, add as 32-bit integers
\end{lstlisting}

The integer suffix can have a + to indicate that a larger size is allowed, for example:
\begin{lstlisting}
   r0 = add.16 (r0, 5)  // r0 += 2, result is 16 bit
   r0 = add.16+(r0, 5)  // r0 += 2, result is at least 16 bit
\end{lstlisting}

The add.16 instruction will make a 16-bit addition and make sure that bit 16-63 of r0 will be zero. add.16+ allows the assembler to use a larger integer size if we don't care what the value of bit 16-63 is. In this it can use a tiny instruction with 64-bit integer size instead of a full-size instruction with 16-bit integer size.

\vspace{2mm}
Memory operands are enclosed in square brackets. Additional options are indicates as name=value. For example:

\begin{lstlisting}
   v0 = add.f (v1, [r2+0x100], length=r3, mask=v4, fallback=v1)
   v0 = add.d (v1, [r2+16], broadcast=r3)
   r1 = add.64(r2, [r3+r4*8], limit=20)
\end{lstlisting}
Here, length=r3 means that the memory operand at address [r2+0x100] is r3 bytes long. mask=v4 means that register v4 is used as mask. fallback=v1 means that the output will be taken from v1 for elements where the mask is 0. broadcast=r3 means that the scalar memory operand is broadcast to a vector of r3 bytes (or r3/8 elements since the size of a double is 8 bytes). limit=20 means that the array pointed to by r3 has 20 elements and a trap has to be generated if the index r4 is above 20.

\vspace{2mm}
Operations with conditional jump are written with the jump after the arithmetic operation:
\begin{lstlisting}
   r1 = sub.32+ (r1, 1), jump_pos LABEL1
\end{lstlisting}
This will subtract 1 from r1 as an integer with at least 32 bits, and jump to LABEL1 if the result, interpreted as a signed integer, is positive.

\vspace{2mm}
Traditional assemblers often have metaprogramming features such as macros, preprocessing conditionals and preprocessing loops. The syntaxes used for these features look like awkward ad hoc solutions with no overall logical structure. We would prefer a syntax that makes a clear distinction between metaprogramming and regular assembly code. The metaprogramming syntax should support integer and floating point variables, strings, macros, conditionals and loops in a way that resembles a structured programming language. 

\end{document}
