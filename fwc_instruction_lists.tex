% chapter included in forwardcom.tex
\documentclass[forwardcom.tex]{subfiles}
\begin{document}
\RaggedRight

\chapter{Instruction lists}
The ForwardCom instructions are listed in a comma-separated file instruction\_list.csv. This file is intended for use by assemblers, disassemblers, debuggers and emulators. The list is preliminary and subject to possible changes. Please remember to keep the lists in this document and the list in the instruction\_list.cvs file synchronized.
\vspace{2mm}

The instruction list file has the following fields:

\begin{longtable} {|p{18mm}|p{100mm}|}
\caption{Fields in instruction list file} 
\label{table:fieldsInInstructionListFile}
\\
\endfirsthead
\endhead
\hline
\bfseries Field & \bfseries Meaning  \\
\hline
Name & Name of instruction as used by assembler.  \\
\hline
Category & 1: single format instruction, \newline
           2: tiny instruction,  \newline
           3: multi-format instruction,  \newline
           4: jump instruction. \\
\hline
Formats & See table \ref{table:MeaningOfFormatsFieldInInstructionListFile} below.  \\
\hline
Template & Hexadecimal number:  \newline
           0xA - 0xE for template A - E,  \newline
           0x1 for tiny template,  \newline
           0x0 for multiple templates. \\
\hline
Source operands & Number of source operands, including register, memory and immediate operands, but not including mask, option bits, vector length, and index. \\
\hline
OP1 & Operation code OP1. \\
\hline
OP2 & Additional operation code OP2. Zero if none. \\
\hline
OP3 bits used & Number of bits of OP3 field used for options. OP3 is used for shift count in format 2.5 and 2.9 only if the value specified here is zero. \\
\hline
Operand types general purpose registers & Hexadecimal number indicating required and optional support for each operand type with general purpose registers. See table \ref{table:OperandTypesInInstructionList} below for meaning of each bit. \\
\hline
Operand types scalar & Hexadecimal number indicating required and optional support for each operand type for scalar operations in vector registers. See table \ref{table:OperandTypesInInstructionList} below for meaning of each bit. \\
\hline
Operand types vector & Hexadecimal number indicating required and optional support for each operand type for vector operations. See table \ref{table:OperandTypesInInstructionList} below for meaning of each bit. \\
\hline
Immediate operand type & Type of immediate operand for single-format instructions. See table \ref{table:immediateOperantTypesInInstructionList} below. \\
\hline
Description & Description of the instruction and comments. \\
\hline
\end{longtable}

\pagebreak % The text in the multirow box below disappears if there is a page break in it. 
           % Put page break here instead to prevent this
\label{table_format_field_in_list}
\begin{longtable} {|p{18mm}|p{14mm} p{80mm}|}
\caption{Meaning of formats field in instruction list file} 
\label{table:MeaningOfFormatsFieldInInstructionListFile}
\\
\endfirsthead
\endhead
\hline
\bfseries Category & \multicolumn{2}{|l|}{\bfseries Interpretation of formats field} \\
\hline
1.  Single format instruction & \multicolumn{2}{|p{102mm}|}{
Number with three hexadecimal digits. \newline
The leftmost digit is the value of the IL field (0-3). \newline
The middle digit is he value of mode field or the combined M+mode field (0-9).\newline
The rightmost digit is the sub-mode defined by OP3 in mode 2.4.x and 2.8.x or OP1 in mode 2.7.x. Zero otherwise. \newline
For example 0x283 means format 2.8.3.
}  \\
\hline
\multirow{11}{*}{\parbox{18mm}{2. Tiny instruction}} 
  &  0 & no operands. \\
  &  1 & RD = general purpose destination register, RS = immediate operand. \\
  &  2 & RD = g. p. destination register, RS = g. p. source register. \\
  &  4 & RD = g. p. destination register, RS = pointer to memory source operand. \\
  &  5 & RD = g. p. source register, RS = pointer to memory destination operand. \\
  &  8 & RD = vector destination register, RS unused. \\
  &  9 & RD = vector destination register, RS immediate operand. \\
  & 10 & RD = vector destination register, RS vector source register. \\
  & 11 & RD = vector source register, RS g. p. destination register r0-r14,r31. \\
  & 12 & RD = vector destination register, RS = pointer to memory source operand. \\
  & 13 & RD = vector source register, RS = pointer to memory destination operand. \\
\hline
% \pagebreak % The text in the multirow box disappears if there is a page break in it. 
           % Put pagebreak here to prevent this, or before the table
% \hline
\multirow{27}{*}{\parbox[t]{18mm}{3. Multi-format instruction}} 
  &  \multicolumn{2}{|l|}{
     Hexadecimal number composed of one bit for each format supported:} \\
  &  0x0000001 & Format 0.0: three general purpose registers. \\
  &  0x0000002 & Format 0.1: two general purpose registers, 8-bit immediate. \\
  &  0x0000004 & Format 0.2: Three vector registers. \\
  &  0x0000008 & Format 0.3: Two vectors, 8-bit immediate. \\
  &  0x0000010 & Format 0.4: One vector, memory operand. \\
  &  0x0000020 & Format 0.5: One vector, memory operand with negative index. \\
  &  0x0000040 & Format 0.6: One vector, scalar memory operand with index. \\
  &  0x0000080 & Format 0.7: One vector, scalar memory operand with 8-bit offset. \\
  &  0x0000100 & Format 0.8: One g. p. register, memory operand with index. \\
  &  0x0000200 & Format 0.9: One g. p. register, memory operand with 8-bit offset. \\
  &  0x0000400 & Format 2.0: Two g. p. registers, memory op. with 32-bit offset. \\
  &  0x0000800 & Format 2.1: Three g. p. registers, 32-bit immediate. \\
  &  0x0001000 & Format 2.2: One vector register, memory op. with 32-bit offset. \\
  &  0x0002000 & Format 2.3: Three vector registers, 32-bit immediate. \\
  &  0x0004000 & Format 2.4.0: Two vector reg., scalar memory op. w. 16-bit offset. \\
  &  0x0008000 & Format 2.4.1: Two vector reg., memory op. with 16-bit offset. \\
  &  0x0010000 & Format 2.4.2: Two vector reg., memory op. with negative index. \\
  &  0x0020000 & Format 2.4.3: Two vector reg., scalar mem. op. index and limit. \\
  &  0x0040000 & Format 2.5: Three vector reg., shifted 16-bit immediate. \\
  &  0x0080000 & Format 2.8.0: Three g. p. reg., memory op. with 16-bit offset. \\
  &  0x0100000 & Format 2.8.1: Two g. p. reg., memory op. with unscaled index. \\
  &  0x0200000 & Format 2.8.2: Two g. p. reg., memory op. with scaled index. \\
  &  0x0400000 & Format 2.8.3: Two g. p. reg., memory op. with index and limit. \\
  &  0x0800000 & Format 2.9: Three g. p. reg., shifted 16-bit immediate. \\
  &  0x1000000 & Format 3.1: Three g. p. registers, 64-bit imm. (optional). \\
  &  0x2000000 & Format 3.3: Three vector registers, 64-bit imm. (optional). \\
\hline
\multirow{12}{*}{\parbox[t]{18mm}{4. Jump instruction}} 
  &  \multicolumn{2}{|l|}{
     Hexadecimal number composed of one bit for each format supported:} \\
  &  0x001 & Format 1.4: Two registers, 8-bit offset. \\
  &  0x002 & Format 1.5 C: One register, 8-bit immediate, 8-bit offset. \\
  &  0x004 & Format 1.5 C: 16-bit offset. \\
  &  0x008 & Format 1.5 D: No register, 24-bit offset. \\
  &  0x010 & Format 2.7.0: Two registers, 32-bit offset. \\
  &  0x020 & Format 2.7.1: Two registers, 16-bit immediate, 16-bit offset. \\
  &  0x040 & Format 2.7.2: One register, 8-bit immediate, 32-bit offset. \\
  &  0x080 & Format 2.7.3: One register, 32-bit immediate, 8-bit offset. \\
  &  0x100 & Format 2.7.4: System call, 16-bit function, 32-bit module. \\
  &  0x200 & Format 3.0.1: Two registers, 32-bit immediate, 32-bit offset. \\
  &  0x400 & Format 3.0.1: 64-bit absolute address. \\
\hline
\end{longtable}

\begin{longtable} {|p{18mm}|p{100mm}|}
\caption{
Indication of operand types supported for general purpose registers, scalars in vector
registers, or vectors. The value is a hexadecimal number composed of one bit for each operand
type supported} 
\label{table:OperandTypesInInstructionList} \\
\endfirsthead
\endhead
\hline
0x0001 & 8-bit integer supported. \\
0x0002 & 16-bit integer supported. \\
0x0004 & 32-bit integer supported. \\
0x0008 & 64-bit integer supported. \\
0x0010 & 128-bit integer supported. \\
0x0020 & single precision floating point supported. \\
0x0040 & double precision floating point supported. \\
0x0080 & quadruple precision floating point supported. \\
0x0100 & 8-bit integer optionally supported. \\
0x0200 & 16-bit integer optionally supported. \\
0x0400 & 32-bit integer optionally supported. \\
0x0800 & 64-bit integer optionally supported. \\
0x1000 & 128-bit integer optionally supported. \\
0x2000 & single precision floating point optionally supported. \\
0x4000 & double precision floating point optionally supported. \\
0x8000 & quadruple precision floating point optionally supported. \\
\hline
\end{longtable}

\begin{longtable} {|p{18mm}|p{100mm}|}
\caption{
Immediate operand type for single-format instructions} 
\label{table:immediateOperantTypesInInstructionList}
\\
\endfirsthead
\endhead
\hline
0 & none or multi-format. \\
1 & 4-bit signed integer. \\
2 & 8-bit signed integer. \\
3 & 16-bit signed integer. \\
4 & 32-bit signed integer. \\
5 & 64-bit signed integer. \\
6 & 8-bit signed integer shifted by specified count. \\
7 & 16-bit signed integer shifted by specified count. \\
8 & 16-bit signed integer shifted by 16. \\
9 & 32-bit signed integer shifted by 32. \\
17 & 4-bit unsigned integer. \\
18 & 8-bit unsigned integer. \\
19 & 16-bit unsigned integer. \\
20 & 32-bit unsigned integer. \\
21 & 64-bit unsigned integer. \\
33 & 4-bit signed integer converted to float. \\
34 & 8-bit signed integer converted to float. \\
35 & 16-bit signed integer converted to float. \\
39 & 16-bit signed integer shifted by specified count, converted to float. \\
64 & half precision floating point. \\
65 & single precision floating point. \\
66 & double precision floating point. \\
\hline
\end{longtable}

Jump instructions are listed on page \pageref{table:controlTransferInstructions}. All other categories of instructions are listed in the following tables.

\section{List of multi-format instructions}
The following list covers general instructions that can be coded in most or all of the formats
assigned to multi-format instructions.

\begin{longtable} {|p{18mm}|p{9mm}|p{9mm}|p{76mm}|}
\caption{
List of multi-format instructions} 
\label{table:ListOfMultiFormatInstructions} \\
\endfirsthead
\endhead
\hline
\bfseries Instruction & \bfseries OP1 & \bfseries Source ope-rands & \bfseries Description \\
\hline
nop          &  0 & 0 & No operation. \\
move         &  1 & 1 & Copy value. \\
store        &  2 & 1 & Store value to memory. \\
prefetch     &  3 & 1 & Prefetch from memory. \\
sign\_extend &  4 & 1 & Sign-extend smaller integer to 64 bits. \\
add          &  8 & 2 & src1 + src2. \\
sub          &  9 & 2 & src1 - src2. \\
sub\_r       & 10 & 2 & src2 - src1. \\
compare      & 11 & 2 & Compare. Uses condition codes, see p. \pageref{table:conditionCodesForCompareInstruction}. \\
mul          & 12 & 2 & src1 $\cdot$ src2. \\ 
mul\_hi\_s   & 13 & 2 & (src1 $\cdot$ src2) \textgreater\textgreater{} OS, signed (integer only). \\
mul\_hi\_u   & 14 & 2 & (src1 $\cdot$ src2) \textgreater\textgreater{}  OS, unsigned (integer only). \\
mul\_ex\_s   & 15 & 2 & Multiply even-numbered signed integer vector elements to double size result. \\
mul\_ex\_u   & 16 & 2 & Multiply even-numbered unsigned integer vector elements to double size result. \\
div          & 17 & 2 & src1 / src2 (optional for integer vectors). \\
rem          & 18 & 2 & Modulo (optional for integer vectors). \\
min          & 20 & 2 & Signed minimum. \\
max          & 21 & 2 & Signed maximum. \\
min\_u       & 22 & 2 & Minimum. unsigned for integers, abs for f.p. \\
max\_u       & 23 & 2 & Maximum. unsigned for integers, abs for f.p. \\
and          & 32 & 2 & src1 \& src2. \\
and\_not     & 33 & 2 & src1 \& (\~{}src2). \\
or           & 34 & 2 & src1 \textbar{} src2. \\
xor          & 35 & 2 & src1 \^{} src2. \\
shift\_left  & 36 & 2 & src1 \textless\textless{} src2.  \\
shift\_rightu& 37 & 2 & src1 \textgreater\textgreater{} src2, zero extended. \\
shift\_rights& 38 & 2 & src1 \textgreater\textgreater{} src2, sign extended. \\
rotate       & 39 & 2 & Rotate left if src2 positive, right if negative. \\
extract\_bit & 40 & 2 & Extract bit. (src1 \textgreater\textgreater{} src2) \& 1. \\
set\_bit     & 41 & 2 & Set bit. src1 \textbar{} (1 \textless\textless{} src2). \\
clear\_bit   & 42 & 2 & Clear bit. src1 \& \~{} (1 \textless\textless{} src2). \\
toggle\_bit  & 43 & 2 & Toggle bit. src1 \^{} (1 \textless\textless{} src2). \\
mul\_add     & 46 & 3 & $\pm$ src1 $\pm$ src2 $\cdot$ src3 (optional). \\
add\_add     & 47 & 3 & $\pm$ src1 $\pm$ src2 $\pm$ src3 (optional). \\
userdef55 - userdef62
             & 55-62 & 2 & Reserved for user-defined instructions. \\
undef        & 63 & 2 & Undefined code. Guaranteed to generate trap in all future implementations. \\
\hline
\end{longtable}


\section{List of tiny instructions}
Tiny instructions are fitted two in one 32-bit code word. If a tiny instruction cannot be paired with anything else, it must be paired with a tiny nop.
\vspace{2mm}

Tiny instructions have an operand size of 64 bits unless otherwise noted. RD is the destination register, and in most cases also the first source register. RS can be a register r0-r15, v0-v15, or an immediate sign-extended 4-bit constant. Instructions with a pointer in RS use register r0-r14 as pointer when RS is 0-14, and the stack pointer (r31) when RS is 15.
\vspace{2mm}

It is not possible to jump to the second instruction in a tiny pair because instruction addresses must be divisible by four. If an interrupt or trap occurs in a tiny instruction then the interrupt handler must remember which of the two tiny instructions in the pair was interrupted.

\begin{longtable} {|p{20mm}|p{8mm}|p{80mm}|}
\caption{List of tiny instructions with general purpose registers} 
\label{table:tinyInstructionsGP} \\
\endfirsthead
\endhead
\hline
\bfseries Instruction & \bfseries OP1 & \bfseries Description \\
\hline
nop           &  0 & No operation. \\
move          &  1 & RD = sign-extended constant RS. \\
add           &  2 & RD += sign-extended constant RS. \\
sub           &  3 & RD --= sign-extended constant RS. \\
shift\_left   &  4 & RD \textless\textless{}= unsigned constant RS. \\
shift\_rightu &  5 & RD \textgreater\textgreater{}= unsigned constant RS (zero extended). \\
move          &  8 & RD = register operand RS. \\
add           &  9 & RD += register operand RS. \\
sub           & 10 & RD --= register operand RS. \\
and           & 11 & RD \&= register operand RS. \\
or            & 12 & RD \textbar= register operand RS. \\
xor           & 13 & RD \^{}= register operand RS. \\
move          & 14 & Read RD from memory operand with pointer RS (RS = r0-r14, r31). \\
store         & 15 & Write RD to memory operand with pointer RS (RS = r0-r14, r31). \\
\hline
\end{longtable}

\begin{longtable} {|p{20mm}|p{8mm}|p{80mm}|}
\caption{List of tiny instructions with vector registers}
\label{table:tinyInstructionsVector}  \\
\endfirsthead
\endhead
\hline
\bfseries Instruction & \bfseries OP1 & \bfseries Description \\
\hline
clear         & 16 & Clear register RD by setting the length to zero. \\
move          & 17 & RD = signed 4-bit integer RS, converted to single precision scalar. \\
move          & 18 & RD = signed 4-bit integer RS, converted to double precision scalar. \\
move          & 19 & RD = RS. Copy vector of any type. \\
add           & 20 & RD += RS, single precision float vector. \\
add           & 21 & RD += RS, double precision float vector. \\
sub           & 22 & RD --= RS, single precision float vector. \\
sub           & 23 & RD --= RS, double precision float vector. \\
mul           & 24 & RD *= RS, single precision float vector. \\
mul           & 25 & RD *= RS, double precision float vector. \\
add\_cps      & 28 & Get size of compressed image for RD and add it to general purpose register RS. \\
sub\_cps      & 29 & Get size of compressed image for RD and subtract it from general purpose register RS. \\
restore\_cp   & 30 & Restore vector register RD from compressed image pointed to by RS. \\
save\_cp      & 31 & Save vector register RD to compressed image pointed to by RS. \\
\hline
\end{longtable}

\section{List of single-format instructions}
These instructions are mostly available in only one or a few formats.

\begin{longtable} {|p{20mm}|p{10mm}|p{8mm}|p{75mm}|}
\caption{List of single-format instructions with general purpose registers} 
\label{table:ListOfSingleFormatInstructionsGP} \\
\endfirsthead
\endhead
\hline
\bfseries Instruction & \bfseries Format &\bfseries OP1 & \bfseries Description \\
\hline
bitscan\_f    & 1.0   &  1 & Bit scan forward. Find index to lowest set bit in RS (optional). \\
bitscan\_r    & 1.0   &  2 & Bit scan reverse. Find index to highest set bit in RS (optional). \\
round\_d2     & 1.0   &  3 & Round down RS to nearest power of 2. \\
round\_u2     & 1.0   &  4 & Round up RS to nearest power of 2. \\
move          & 1.1   &  0 & Move 16-bit sign-extended constant to general purpose register RD. \\
move\_u       & 1.1   &  1 & Move 16-bit zero-extended constant to general purpose register RD
(can be used as first step of loading a 32-bit constant if double size
instructions are not supported). \\
add           & 1.1   &  2  & Add 16-bit sign-extended constant to RD. \\
sub           & 1.1   &  3  & Subtract 16-bit sign-extended constant from RD. \\
subr          & 1.1   &  4  & Subtract RD from 16-bit sign-extended constant. \\
mul           & 1.1   &  5  & Multiply RD with 16-bit sign-extended constant. \\
div           & 1.1   &  6  & Divide RD with 16-bit sign-extended constant. \\
add           & 1.1   &  7  & Shift 16-bit signed constant left by 16 and add to RD. \\
move          & 1.1   & 16  & RD = IM2 \textless\textless{} IM1. Sign-extend IM2 to 64 bits and shift left by the unsigned value IM1. \\
add           & 1.1   & 17  & RD += IM2 \textless\textless{} IM1. Sign-extend IM2 to 64 bits, shift left by the unsigned value IM1, add to RD. \\
and           & 1.1   & 18  & RD \&= IM2 \textless\textless{} IM1. Sign-extend IM2 to 64 bits, shift left by the unsigned value IM1, AND with RD. \\
or            & 1.1   & 19  & RD \textbar{}= IM2 \textless\textless{} IM1. Sign-extend IM2 to 64 bits, shift left by the unsigned value IM1, OR with RD. \\
xor           & 1.1   & 20  & RD \^{}= IM2 \textless\textless{} IM1. Sign-extend IM2 to 64 bits, shift left by the unsigned value IM1, XOR with RD. \\
abs           & 1.8   &  0  & Absolute value of integer. Use saturation if IM1 = 1. \\
shift\_add    & 1.8   &  1  & Shift and add. RD += RS \textless\textless{} IM1 (shift right zero extended if IM1 negative). \\
read\_spe     & 1.8   & 32  & Read special register RS into g. p. register RD. \\
write\_spe    & 1.8   & 33  & Write g. p. register RS to special register RD. \\
read\_cpb     & 1.8   & 34  & Read capabilities register RS into g. p. register RD. \\
write\_cpb    & 1.8   & 35  & Write g. p. register RS to capabilities register RD. \\
read\_perf    & 1.8   & 36  & Read performance counter. \\
read\_perfs   & 1.8   & 37  & Read performance counter, serializing. \\
read\_sys     & 1.8   & 38  & Read system register RS into g. p. register RD. \\
write\_sys    & 1.8   & 39  & Write g. p. register RS to system register RD. \\
load\_hi      & 2.6   &  0  & Load 32-bit constant into the high part of a general purpose register. The low part is zero. RD = IM2 \textless\textless{} 32. \\
insert\_hi    & 2.6   &  1  & Insert 32-bit constant into the high part of a general purpose register, leaving the low part unchanged.
RD = (RS \& 0xFFFFFFFF) \textbar{} (IM2 \textless\textless{} 32). \\
add\_unsigned & 2.6   &  2  & Add zero-extended 32-bit constant to general purpose register. \\
sub\_unsigned & 2.6   &  3  & Subtract zero-extended 32-bit constant from general purpose register. \\
add\_hi       & 2.6   &  4  & Add 32-bit constant to high part of general purpose register. RD = RS + (IM2 \textless\textless{} 32). \\
and\_hi       & 2.6   &  5  & AND high part of general purpose register with 32-bit constant. RD = RS \& (IM2 \textless\textless{} 32). \\
or\_hi        & 2.6   &  6  & OR high part of general purpose register with 32-bit constant. RD = RS \textbar{} (IM2 \textless\textless{} 32). \\
xor\_hi       & 2.6   &  7  & XOR high part of general purpose register with 32-bit constant. RD = RS \^{} (IM2 \textless\textless{} 32). \\
address       & 2.6   & 32  & RD = RS + IM2, RS can be THREADP (28), DATAP (29) or IP (30). \\
\hline
\end{longtable}

\begin{longtable} {|p{20mm}|p{10mm}|p{8mm}|p{75mm}|}
\caption{List of single-format instructions with vector registers and mixed register types} 
\label{table:ListOfSingleFormatInstructionsVector} \\
\endfirsthead
\endhead
\hline
\bfseries Instruction & \bfseries Format &\bfseries OP1, OP2 & \bfseries Description \\
\hline
set\_len      & 1.2 &  0 & RD = vector register RT with length changed to value of RS. \\
get\_len      & 1.2 &  1 & Get length of vector register RS into general purpose register RD. \\
set\_num      & 1.2 &  2 & Change the length of vector register to RS$\cdot$OS. \\
get\_num      & 1.2 &  3 & Get length of vector register divided by the operand size. \\
compress      & 1.2 &  4 & Compress vector RT of length RS to a vector of half the length and half the element size. Double precision $\rightarrow$ single precision, 64-bit
integer $\rightarrow$ 32-bit integer, etc. \\
compress\_ss  & 1.2 &  5 & Compress integer vector RT of length RS to a vector of half the length and half the element size, signed with saturation (optional). \\
compress\_us  & 1.2 &  6 & Compress integer vector RT of length RS to a vector of half the length and half element size, unsigned with saturation (optional). \\
expand        & 1.2 &  7 & Expand vector RT of length RS/2 and half the specified element size to a vector of length RS with the specified element size. Half
precision $\rightarrow$ single precision, 32-bit integer $\rightarrow$ 64-bit integer with sign extension, etc. \\
expand\_us    & 1.2 &  8 & Expand integer vector RT of length RS/2 and half the specified element size to a vector of length RS with the specified element
size. 32-bit integer $\rightarrow$ 64-bit integer with zero extension, etc. \\
compress\_sparse&1.2 &  9 & Compress sparse vector elements indicated by mask bits into contiguous vector. RS = length of input vector. (optional). \\
expand\_sparse& 1.2 & 10 & Expand contiguous vector into sparse vector with positions
indicated by mask bits. RS = length of output vector. (optional). \\
extract       & 1.2 & 11 & Extract one element from vector RT, starting at offset RS$\cdot$OS, with size OS into scalar in vector register RD. \\
insert        & 1.2 & 12 & Replace one element in vector RD, starting at offset RS$\cdot$OS, with scalar RT. \\
broadcast     & 1.2 & 13 & Broadcast first element of vector RT into all elements of RD with length RS. \\
bits2bool     & 1.2 & 14 & The lower n bits of RT are unpacked into a boolean vector RD with length RS, with one bit in each element, where n = RS / OS. \\
bool2bits     & 1.2 & 15 & The boolean vector RT with length RS is packed into the lower n bits of RD, taking bit 0 of each element, where n = RS / OS. The
length of RD is at least sufficient to contain n bits. \\
bool\_reduce  & 1.2 & 16 & The boolean vector RT with length RS is reduced by combining bit 0 of all elements. The output is a scalar integer where bit 0 is the
AND combination of all the bits, and bit 1 is the OR combination of
all the bits. The remaining bits are reserved for future use. \\
shift\_expand & 1.2 & 18 & Shift vector RT up by RS bytes and extend the vector length by RS. The lower RS bytes of RD will be zero. \\
shift\_reduce & 1.2 & 19 & Shift vector RT down RS bytes and reduce the length by RS. The lower RS bytes of RT are lost. \\
shift\_up     & 1.2 & 20 & Shift elements of vector RT up RS elements. The lower RS elements of RD will be zero, the upper RS elements of RT are lost. \\
shift\_down   & 1.2 & 21 & Shift elements of vector RT down RS elements. The upper RS
elements of RD will be zero, the lower RS elements of RT are lost. \\
div\_ex\_s    & 1.2 & 24 & Divide vector of double-size signed integers RS by signed integers RT. RS has element size 2$\cdot$OS. These are divided by the even numbered
elements of RT with size OS. The results are stored in
the even-numbered elements of RD. The remainders are stored in
the odd-numbered elements of RD. (Optional for vectors). \\
div\_ex\_u    & 1.2 & 25 & Same, with unsigned integers. (Optional for vectors). \\
sqrt          & 1.2 & 26 & Square root (floating point, optional). \\
add\_c        & 1.2 & 28 & Add with carry. Vector has two elements. The upper element is used as carry on input and output (optional). \\
sub\_b        & 1.2 & 29 & Subtract with borrow. Vector has two elements. The upper element is used as borrow on input and output (optional). \\
add\_ss       & 1.2 & 30 & Add integer vectors, signed with saturation (optional). \\
add\_us       & 1.2 & 31 & Add integer vectors, unsigned with saturation (optional). \\
sub\_ss       & 1.2 & 32 & Subtract integer vectors, signed with saturation (optional). \\
sub\_us       & 1.2 & 33 & Subtract integer vectors, unsigned with saturation (optional). \\
mul\_ss       & 1.2 & 34 & Multiply integer vectors, signed with saturation (optional). \\
mul\_us       & 1.2 & 35 & Multiply integer vectors, unsigned with saturation (optional). \\
shl\_ss       & 1.2 & 36 & Shift left integer vectors, signed with saturation (optional). \\
shl\_us       & 1.2 & 37 & Shift left integer vectors, unsigned with saturation (optional). \\
add\_oc       & 1.2 & 38 & add with overflow check (optional). \\
sub\_oc       & 1.2 & 39 & subtract with overflow check (optional). \\
subr\_oc      & 1.2 & 40 & subtract reverse with overflow check (optional). \\
mul\_oc       & 1.2 & 41 & multiply with overflow check (optional). \\
div\_oc       & 1.2 & 42 & divide with overflow check (optional). \\
input         & 1.2 & 48 & read from input port. RD = vector register, RT = port address, RS = vector length (privileged instruction). \\
output        & 1.2 & 49 & write to output port. RD = vector register source operand, RT = port address, RS = vector length (privileged instruction). \\
gp2vec        & 1.3 B &  0 & Move value of general purpose register RS to scalar in vector register RD. \\
set\_bits\_x   & 1.3 B &  1 & Set all bits except one. RD = RS \textbar{} \~{} (1 \textless\textless{} IM1). \\
clear\_bits\_x & 1.3 B &  2 & Clear all bits except one. RD = RS \& (1 \textless\textless{} IM1). \\
make\_sequence& 1.3 B &  3 & Make a vector with RS sequential numbers. First value is IM1. \\
mask\_length  & 1.3 B &  4 & Make mask with true in the first RS bytes. Option bits in IM1. \\
vec2gp        & 1.3 B &  8 & Move value of first element of vector register RS to general purpose register RD. \\
bitscan\_f    & 1.3 B &  9 & Bit scan forward. Find index to lowest set bit in RS (optional for vectors). \\
bitscan\_r    & 1.3 B & 10 & Bit scan reverse. Find index to highest set bit in RS (optional for vectors). \\

float2int     & 1.3 B & 12 & Conversion of floating point to integer with the same operand size. The rounding mode is specified in IM1. \\
int2float     & 1.3 B & 13 & Conversion of integer to floating point with same operand size. \\

round         & 1.3 B & 14 & Round floating point to integer in floating point  representation. The rounding mode is specified in IM1. \\
round2n       & 1.3 B & 15 & Round to nearest multiple of $2^n$. \newline 
RD = $2^n\cdot$ round($2^{-n}\cdot$ RS). $n$ is a signed integer constant in IM1 (optional). \\
abs           & 1.3 B & 16 & Absolute value of integer. Uses saturation if IM1 = 1. \\
popcount      & 1.3 B & 17 & Count the number of bits in RS that are 1. \\
broadcast     & 1.3 B & 18 & Broadcast 8-bit constant into all elements of RD with length RS (31 in RS field gives scalar output). \\
fp\_category  & 1.3 B & 19 & Check if floating point numbers belong to the categories indicated by constant. \\
byte\_reverse & 1.3 B & 20 & Reverse the order of bytes in each element of vector. \\
bit\_reverse  & 1.3 B & 21 & Reverse the order of bits in each element of vector (optional). \\
truth\_tab2   & 1.3 B & 24 & Boolean function of two inputs, given by a truth table. \\
read\_spev    & 1.3 B & 30 & Read special register RT into vector register RD with  length RS. \\
move          & 1.3 C & 32 & Move 16 bit constant to 16-bit scalar (optional). \\
add           & 1.3 C & 33 & Add broadcast 16 bit constant to 16-bit vector elements (optional). \\
and           & 1.3 C & 34 & AND broadcast 16 bit constant with 16-bit vector elements (optional). \\
or            & 1.3 C & 35 & OR broadcast 16 bit constant with 16-bit vector elements (optional). \\
xor           & 1.3 C & 36 & XOR broadcast 16 bit constant with 16-bit vector elements (optional). \\
move          & 1.3 C & 38 & RD = IM2 \textless\textless{} IM1. Sign-extend IM2 to 32 bits and shift left by the unsigned value IM1 to make 32 bit scalar (optional). \\
move          & 1.3 C & 39 & RD = IM2 \textless\textless{} IM1. Sign-extend IM2 to 64 bits and shift left by the unsigned value IM1 to make 64 bit scalar (optional). \\
add           & 1.3 C & 40 & RD += IM2 \textless\textless{} IM1. Add broadcast shifted signed constant to 32-bit vector elements (optional). \\
add           & 1.3 C & 41 & RD += IM2 \textless\textless{} IM1. Add broadcast shifted signed constant to 64-bit vector elements (optional). \\
and           & 1.3 C & 42 & RD \&= IM2 \textless\textless{} IM1. AND broadcast shifted signed constant with 32-bit vector elements (optional). \\
and           & 1.3 C & 43 & RD \&= IM2 \textless\textless{} IM1. AND broadcast shifted signed constant with 64-bit vector elements (optional). \\
or            & 1.3 C & 44 & RD \textbar{}= IM2 \textless\textless{} IM1. OR broadcast shifted signed constant with 32-bit vector elements (optional). \\
or            & 1.3 C & 45 & RD \textbar{}= IM2 \textless\textless{} IM1. OR broadcast shifted signed constant with 64-bit vector elements (optional). \\
xor           & 1.3 C & 46 & RD \^{}= IM2 \textless\textless{} IM1. XOR broadcast shifted signed constant with 32-bit vector elements (optional). \\
xor           & 1.3 C & 47 & RD \^{}= IM2 \textless\textless{} IM1. XOR broadcast shifted signed constant with 64-bit vector elements (optional). \\
add           & 1.3 C & 48 & RD += IM21 \textless\textless{} 16. Add broadcast signed 16-bit constant shifted left by 16 to 32-bit vector elements (optional). \\
add           & 1.3 C & 49 & RD += IM21 \textless\textless{} 16. Add broadcast signed 16-bit constant shifted left by 16 to 64-bit vector elements (optional). \\
mov           & 1.3 C & 56 & Move converted half precision floating point constant to single
precision scalar (optional). \\
mov           & 1.3 C & 57 & Move converted half precision floating point constant to double
precision scalar (optional). \\
add           & 1.3 C & 58 & Add broadcast half precision floating point constant to single
precision vector (optional). \\
add           & 1.3 C & 59 & Add broadcast half precision floating point constant to double
precision vector (optional). \\
mul           & 1.3 C & 60 & Multiply broadcast half precision floating point constant with single precision vector (optional). \\
mul           & 1.3 C & 61 & Multiply broadcast half precision floating point constant with double precision vector (optional). \\

concatenate   & 2.5 & 2, 8 & A vector RT of length RS and a vector RU of length RS are
concatenated into a vector RD of length 2$\cdot$RS. \\

permute       & 2.5 & 2, 9 & The vector elements of RT are permuted within each block of size RS bytes, using indices in RU. Each index is relative to the
beginning of a block. An index out of range produces zero. The
maximum block size is implementation dependent. \\

interleave    & 2.5 & 2,10 & Interleave elements of vectors RT and RU to produce vector RD of length RS. Even-numbered elements of the destination come from RT and odd-numbered elements from RU. (optional). \\

truth\_tab3   & 2.5 & 3, 8 & Boolean function of three inputs, given by a truth table (optional). \\
truth\_tab4   & 2.5 & 4, 8 & Boolean function of four inputs, given by a truth table (optional). \\
mul\_add      & 2.5 & 3, 9 & RD = $\pm$ RS $\pm$ RT $\cdot$ RU (optional but recommended). \\
add\_add      & 2.5 & 3,10 & RD = $\pm$ RS $\pm$ RT $\pm$ RU (optional). \\
add\_add\_add & 2.5 & 3,11 & RD = $\pm$ RS $\pm$ RT $\pm$ RU $\pm$ IM2 
Add three vector register operands and a 16-bit constant IM2 (optional). \\
add\_add\_add & 2.5 & 4,11 & RD = $\pm$ RD $\pm$ RS $\pm$ RT $\pm$ RU
Add four vector register operands (optional). \\

repeat\_block  & 2.5 & 8, 8 & Repeat a block of data to make a longer vector. RT is input vector containing data block to repeat. RU is length of block to repeat (g. p. register). RS is length of destination vector RD. (optional. RU must be a multiple of 4). \\

repeat\_within \_blocks & 2.5 & 8, 9 & Broadcast the first element of each block of data in a vector to the entire block. RT is input vector containing data blocks. RU is length of each block (g. p. register). RS is length of destination vector RD. The operand size must be at least 4 bytes. (optional). \\

load\_hi      & 2.7 & 16 & Make vector of two elements. dest[0] = 0, dest[1] = IM2. \\
insert\_hi    & 2.7 & 17 & Make vector of two elements. dest[0] = src1[0], dest[1] = IM2. \\
make\_mask    & 2.7 & 18 & Make vector where bit 0 of each element comes from bits in IM2, the remaining bits come from RS. \\
replace       & 2.7 & 19 & Replace elements in RS by constant IM2. \\
replace\_even & 2.7 & 20 & Replace even-numbered elements in RS by constant
IM2. \\
replace\_odd  & 2.7 & 21 & Replace odd-numbered elements in RS by constant
IM2. \\
broadcast     & 2.7 & 22 & Broadcast 32-bit constant into all elements of RD with length RS (31 in RS field gives scalar output). \\
permute       & 2.7 & 33 & The vector elements of RT are permuted within each block of size RS bytes. The 4$\cdot$n bits of IM2 are used as index with 4 bits for
each element in blocks of size n. The same pattern is used in each
block. The number of elements in each block, n = RS / OS $\leq$ 8. \\
\hline
\end{longtable}

\begin{longtable} {|p{20mm}|p{10mm}|p{8mm}|p{75mm}|}
\caption{List of single-format instructions with memory operands.} 
\label{table:ListOfSingleFormatInstructionsMemory} \\
\endfirsthead
\endhead
\hline
\bfseries Instruction & \bfseries Format &\bfseries OP1, OP2 & \bfseries Description \\
\hline
store         & 2.7 B & 48 & Store 32-bit constant IM2 to memory operand with base RT and 8-bit offset IM1 (optional). \\
fence         & 2.4.x & 0, 8 & Memory fence. read, write or full indicated by OP3.  \\
cmp\_swap     & 2.8.x & 1, 8 & Atomic compare and exchange.  \\
read\_insert  & 2.4.0 2.4.3 & 2, 8 & Replace one element in vector RD, starting at offset 
RS$\cdot$OS, with scalar memory operand (optional).  \\
move\_store   & 2.4.x & 3, 8 & Conditional move and store.\newline
                               Mask bits = 01 or 11: store RU.\newline 
                               Mask bits = 10: store zero. \newline
                               Mask bits = 11: store RD. \newline
                               (optional). \\
extract\_store& 2.4.0 & 3, 9 & Extract one element from vector RD, starting at offset RS$\cdot$OS, with size OS into memory operand with base RT and offset IM2 (optional). \\
extract\_store& 2.4.3 & 3, 9 & Extract one element from vector RD, starting at offset RS$\cdot$OS, with size OS into memory operand with base RT, scaled index RU and unsigned limit RU $\leq$ IM2 (optional). \\
compress\_store&2.4.1 & 3, 10 & Compress vector RD of length RS to a vector of half the length and half the element size. Double precision $\rightarrow$ single precision, 64-bit integer $\rightarrow$ 32-bit integer, etc. Store at memory with base RT, offset IM2, length RS/2 (optional). \\
add\_store    & 2.4.x & 4, 8 & Add RD and RU, store the result to memory operand (optional). \\
sub\_store    & 2.4.x & 4, 9 & Subtract RU from RD, store the result to memory operand (optional). \\
mul\_store    & 2.4.x & 4, 10 & Multiply RD and RU, store the result to memory operand (optional). \\
read\_memory\_\newline map  & 2.4.2 & 48, 8 & Read memory map. RD = map entry, RT = memory pointer, RS = vector length and negative index to both source and destination (privileged). \\
write\_memory\_\newline map & 2.4.2 & 48, 9 & Write memory map. RD = map entry, RT = memory pointer, RS = vector length and negative index to both source and destination (privileged). \\

\hline
\end{longtable}

\end{document}
