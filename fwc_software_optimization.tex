% chapter included in forwardcom.tex
\documentclass[forwardcom.tex]{subfiles}
\begin{document}
\RaggedRight

\chapter{Software optimization guidelines} \label{SoftwareOptimization}

The ForwardCom system is designed with high performance as a top priority. The following guidelines may help programmers and compiler makers obtain optimal performance.

\subsubsection{Minimize instruction size}
The assembler should automatically pick the smallest possible version of an instruction. Where the instruction contains an immediate constant, you may use a small sign-extended integer, a shifted integer, or a half-precision floating point constant if it fits the actual value of the constant. Jump, call and branch instructions within the same compiled module should use the smallest offset that fits the actual distance to the target. Jumps and calls to external modules need relocation at the link stage so that the actual distance is not known when the module is assembled. You may use 16, 24 or 32 bit scaled offsets for external jumps and calls depending on the estimated size of the whole program.

\subsubsection{Use moderately complex instructions}
ForwardCom includes instructions that do multiple things in one instruction, but only if it fits into the general pipeline structure. ForwardCom has few or no microcoded instructions because such instructions are likely to have poor performance. Most ForwardCom Instructions have a throughput of at least one instruction per clock cycle regardless of whether they do something simple or something complex.
\vspace{2mm}

Instructions that can do multiple things fast include:

\begin{itemize}
\item Instructions with a memory operand. These instructions can calculate the address of a memory operand, read from memory, and do an arithmetic or logical calculation.

\item ALU and conditional jump. These instructions can make an arithmetic or logical calculation and make a conditional jump depending on the result of the calculation.

\item Add/subtract and unconditional jump. This instruction does two unrelated things. It is implemented just because it fits into the general scheme of ALU and conditional jump with no extra hardware costs.

\item Multiway jump or call. This instruction reads an element from an array of relative jump addresses, calculates the target address and makes a jump or call to this address.

\item Masking and fallback. Most common instructions have options for predicated or masked execution with a specified fallback value if the predicate or mask bit is zero. This feature may be used for replacing a poorly predictable branch. However, if this increases the length of a critical dependency chain and the branch has good prediction, then it may be better to use a branch.

\item Compare and bit test instructions. These instructions can use the mask register and the fallback register as extra boolean operands. This can eliminate any extra instructions for AND'ing or OR'ing the results of multiple compares or bit tests.

\item 3-input instructions: mul\_add, add\_add, and truth\_tab3. These instructions are optional. The hardware should only implement these instructions if they are efficient.

\end{itemize}

\subsubsection{Use immediate constants}
ForwardCom allows you to embed scalar and broadcasted constants in instructions. For example, it is obviously more efficient to have a 32-bit floating point constant embedded in an instruction than to have a 32-bit address of a 32-bit value in static memory. 64-bit or double precision constants may also be embedded in instructions if triple-size instructions are supported by the hardware. If a constant is used inside a critical loop then it may be better to load the constant into a register before the loop.

\subsubsection{Optimize cache use}
Memory and cache throughput is often a bottleneck. You can improve caching 
in several ways:

\begin{itemize}
\item Optimize code caching by minimizing instruction sizes, pairing tiny instructions, and inlining functions.

\item Optimize data caching by embedding immediate constants.

\item Use register variables instead of saving variables in memory.

\item Use registers as function parameters.

\item Use vector registers as arrays instead of storing arrays in memory.

\item Avoid spilling registers to memory by using the information about register use in object files, as described on page \pageref{registerUsageConvention}.

\end{itemize}

\subsubsection{Use the vector loop feature}
Array loops are particularly efficient if the vector loop feature described on page \pageref{vectorLoops} is used. Loops containing function calls can be vectorized if the functions allow vector parameters.

\subsubsection{Avoid long dependency chains}
ForwardCom may be implemented on a superscalar processor that can execute multiple instructions simultaneously. This works most efficiently if the code does not contain long dependency chains.


\end{document}
