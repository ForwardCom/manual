% chapter included in forwardcom.tex
\documentclass[forwardcom.tex]{subfiles}
\begin{document}
\RaggedRight


\chapter{Other implementation details}
\section{Endianness} \label{endianness}
The memory organization is little endian. Instructions for byte swapping are provided for reading and writing big endian binary data files.

\subsubsection{Rationale}
The storage of vectors in memory would depend on the element size if the organization was big endian. Assume, for example, that we have a 128 bit vector register containing four 32-bit integers, named A, B, C, D. With little endian organization, they are stored in memory in the order:
\vspace{2mm}

A0, A1, A2, A3, B0, B1, B2, B3, C0, C1, C2, C3, D0, D1, D2, D3,
\vspace{2mm}

where A0 is the least significant byte of A and D3 is the most significant byte of D. With big endian organization we would have:
\vspace{2mm}

A3, A2, A1, A0, B3, B2, B1, B0, C3, C2, C1, C0, D3, D2, D1, D0.
\vspace{2mm}

This order would change if the same vector register is organized, for example, as eight integers of 16 bits each or two integers of 64 bits each. In other words, we would need different read and write instructions for different vector organizations. 
\vspace{2mm}

Little endian organization is more common for a number of reasons that have been discussed many times elsewhere.
\vspace{2mm}

\section{Implementation of call stack} \label{callStackAlternatives}
There are various methods for saving the return addresses for function calls: a link register, a separate call stack, or a unified stack for return addresses and local data. Here, we will discuss the pro's and con's of each of these methods.
\vspace{2mm}

\subsubsection{Link register}
Some systems use a link register to hold the return address. The advantage of a link register is that a leaf function can be called without storing anything on the stack. This saves cache bandwidth in programs with many leaf function calls. The disadvantage is that every non-leaf function needs to save the link register on a stack before calling another function, and restore the leaf register before returning.
\vspace{2mm}

If we decide to have a link register then it should be a special register, not one of the general purpose registers. A link register does not need to support all the things that a general purpose register can do. If the link register is included as one of the general purpose registers then it will be tempting for a programmer to save it to another register rather than on the stack, and then end the function by jumping to that other register. This will work, of course, but it will interfere with the way returns are predicted. The branch predictor uses a special mechanism for predicting returns, which is different from the mechanism used for predicting other jumps and branches. This mechanism, which is called a return stack buffer, is a small rolling cache that remembers the addresses of the last calls. If a function returns by a jump to another register than the link register then it will use the wrong prediction mechanism, and this will cause severe delays due to misprediction of the subsequent series of returns. The return stack buffer will also be messed up if the link register is used for indirect jumps or other purposes.
\vspace{2mm}

The only instructions that are needed for the link register other than call and return, are push and pop. We can reduce the number of instructions in non-leaf functions by making a combined instruction for ``push link register and then call a function'' which can be used for the first function call in a non-leaf function, and another instruction for ``pop link register and then return'' to end a non-leaf function. However, this will violate the principle that we want to avoid complex instructions in order to simplify the pipeline design.
\vspace{2mm}

The only performance gain we get from using a link register is that it saves cache bandwidth by not saving the return address on leaf function calls. It will not affect performance in applications where cache bandwidth is not a bottleneck. The performance of the return instruction is not influenced by cache bandwidth because it can rely on the prediction in the return stack buffer.
\vspace{2mm}

The disadvantage of using a link register is that the compiler has to treat leaf functions and non-leaf functions differently, and that non-leaf functions need extra instructions for saving and restoring the leaf register on the stack.
\vspace{2mm}

Therefore, we will not use a link register in the ForwardCom architecture.

\subsubsection{Separate call stack} \label{dualStack}
We may have two stacks: 
a call stack for return addresses and a data stack for the local data of each function. A program without recursive functions will usually have a quite limited call depth so that the entire call stack, or at least the ``hot'' part of it, can be stored on the chip. This will improve the performance because no memory or cache operations are needed for call and return operations -- at least not in the critical innermost loops of the program. It will also simplify prediction of return addresses because the on-chip rolling stack and the return stack buffer will be one and the same structure.
\vspace{2mm}

The call stack can be implemented as a rolling register stack on the chip. The call stack is spilled to memory if it overflows. A return instruction after such a spilling event will use the on-chip value rather than the value in memory as long as the on-chip value has not been overwritten by new calls. Therefore, the spilling event is unlikely to occur more than once in the innermost part of a program.
\vspace{2mm}

The pointer for the call stack should not be a general purpose register because the programmer will rarely need to access it directly. Direct manipulation of the call stack is only needed in a stack unroll event (after an exception or long jump) or a task switch.
\vspace{2mm}

A function does not have easy access to the return address that it was called from. Information about the caller may be supplied explicitly as a function parameter in the rare case that it is needed. There is a security advantage in hiding the return address inside the chip. This prevents overwriting return addresses in case of program errors or malicious buffer overflow attacks.
\vspace{2mm}

The disadvantage of having a separate call stack is that it makes memory management more complicated because there are two stacks that can potentially overflow. The size of the call stack can be predicted accurately for programs without recursive functions by using the method described on page \pageref{predictingStackSize}.
\vspace{2mm}

A separate call stack may be implemented with the ForwardCom architecture. The size of the on-chip stack buffer and other details will be implementation-dependent.

\subsubsection{Unified stack for return addresses and local data} \label{singleStack}
Many current systems use the same stack for return addresses and local data. This method may be used with the ForwardCom architecture because it is simple to implement. 

\subsubsection{Conclusion for ForwardCom}
A ForwardCom system may use a separate call stack or a unified stack, but not a link register. The hardware implementation of call and return instructions depends on whether there is one or two stacks. The dual stack system will be used for large processors where performance or security is important, while the unified stack system may be used in small processors where simplicity is preferred. A ForwardCom microprocessor does not have to support both systems, but the software does. 
The calling conventions defined on page \pageref{functionCallingConventions} will make the software compatible with both single stack and dual stack processors.
Tail calls can be implemented efficiently with a simple jump instruction regardless of the stack type.
\vspace{2mm}

\section{Floating point errors and exceptions}
Exceptions for floating point errors are disabled by default, but can be enabled with bits 26-29 in the numeric control register or a mask register. Enabled exceptions are caught as traps (synchronous interrupts).
\vspace{2mm}

It is a problem that an exception caused by a single element in a vector will interrupt the processing of the whole vector. The behavior of a program using floating point vectors will depend on the vector length in case of traps caused by a single vector element. We can rely on the generation and propagation of NAN and INF values instead of traps if we want consistent results on different processors with different vector lengths. 
\vspace{2mm}

\label{nanPropagation}
NAN values will be propagated through the sequence of floating point calculations. A NAN can contain a bit pattern of diagnostic information called the payload, and this bit pattern is propagated to the result. A problem arises when two different NANs are combined, for example NAN1 + NAN2. The IEEE standard (754-2008) specifies that only one of the two NAN operands is propagated to the result. This violates the fundamental principle that addition is commutative. The result can be inconsistent when a compiler swaps the two operands. Another problem with the IEEE standard is that NAN values are not propagated through the max and min instructions according to this standard.
\vspace{2mm}

Here, it is proposed to deviate from this unfortunate standard and output the OR combination of the input NAN payloads when multiple NAN operands are combined. This will make the propagation of NANs more useful and consistent. Different bits in the NAN payload can be used for indicating different error conditions. If multiple different error conditions have arisen in a sequence of calculations then all these conditions can be traced in the final result. This better propagation of NAN values is enabled by setting bit 21 in the numeric control register or in a mask register.
\vspace{2mm}

The implementation will use only one bit in the NAN payload for each error condition. A quiet NAN has bit number -1 of the significand set, while the remaining bits are available for any payload information. The ForwardCom processor puts diagnostic information in the payload if better NAN propagation is enabled by bit 21 in the numeric control register or a mask register. Bit number -2 in the significand indicates invalid arithmetic operations such as 0/0, $0\cdot\infty$, $\infty-\infty$, etc. Bit number -3 indicates a square root of a negative number, and other complex number results. The remaining payload bits are available for other purposes such as function libraries.
\vspace{2mm}

Other methods for generating error messages in function libraries are discussed on page \pageref{errorMessageHandling}.
\vspace{2mm}


\section{Detecting integer overflow} 
\label{integerOverflowDetection}
There is no common standard method for detecting overflow in integer calculations. The detection of overflow in signed integer operations is a real nightmare in some programming languages like C++ (see e. g. 
\href{http://stackoverflow.com/questions/199333/how-to-detect-integer-overflow-in-c-c}{stackoverflow.com/questions/199333/how-to-detect-integer-overflow-in-c-c}).
\vspace{2mm}

It would be nice to have a reliable way of detecting integer overflow and perhaps to propagate it through a series of calculations, analogous to the NAN propagation for floating point calculations, so that errors can be checked at the end of a series of calculations rather than after each operation. Compilers could support this method by offering overflow detection with a try/catch block. It is more likely that compilers will support integer overflow detection if the hardware offers a reasonable method.
\vspace{2mm}

The following methods have been proposed:

\begin{enumerate}
\item Use a few vacant bits in the mask registers for detecting and propagating overflow and other errors. This method has a number of problems that will impede out-of-order execution. The mask register will be used not only for input to each instruction but also output. Each instruction will then have two outputs rather than one. This will make the out-of-order scheduler much more complicated, and it will cause undesired dependencies when the same mask register is used for multiple instructions that otherwise would be independent.

\item Use the even-numbered elements in a vector register for normal calculation on integers 
and use the following odd-numbered elements for the overflow information. The overflow
information is propagated together with the calculated values. 
This method will be efficient for scalar integer calculations, but wasteful for vectors because half the vector elements are used only for this purpose. 

\item Use one element of a vector for the overflow bits of all the other elements. This method may be tempting because it does not waste as much register space as the previous method, but it will have inferior performance because of the transport delay when moving overflow bits to a distant part of a long vector.

\item Add extra bits in the vector registers for overflow information. All vector registers will have one extra overflow bit for each 32 bits of normal data. These overflow bits are preserved when a vector register is saved and restored with the save\_cp and restore\_cp instructions, but they are lost when the vector is saved as normal data. The behavior of the overflow bits is controlled by bits in the numeric control register or a mask register to enable the detection and propagation of signed and unsigned integer overflow.
An extra instruction must be provided for extracting the overflow bits from a vector register.

\item Generate a trap in case of integer overflow. Use a mask register or the numeric control register to control this behavior. Bit 6 enables a trap for unsigned integer overflow and bit 7 enables a trap for signed integer overflow. This method requires little extra code, but it is subject to the problem that the behavior of vector code depends on the vector length in case of traps, as explained in the previous chapter for floating point errors.
\end{enumerate}

Method 2 is tentatively supported here with the optional instructions add\_oc, etc., described on page \pageref{instructionsWithOverflowCheck}. 
\vspace{2mm}

Support for method 4 may be considered, since it would be more efficient and useful. The cost of implementing method 4 is that we will need 3\% more bits in the vector registers; the save\_cp and restore\_cp instructions will be more complicated; and the compiler has to check for overflow before saving vectors to memory in the normal way. 
\vspace{2mm}

Method 5 should be supported. It is useful for integer code in general purpose registers and it is useful for verifying that overflow does not occur in vector registers.
\vspace{2mm}

These methods should not detect overflow in saturated arithmetic instructions and shift instructions.

\section{Multithreading}
The ForwardCom design makes it possible to implement very large vector registers to process large data sets. However, there are practical limits to how much you can speed up the performance by using larger vectors. First, the actual data structures and algorithms often limit the vector length that can be used. And second, large vectors mean longer physical distances on the semiconductor chip and longer transport delays.
\vspace{2mm}

Additional parallelism can be obtained by running multiple threads in each their CPU core. The design should allow multiple CPU chips or multiple CPU cores on the same physical chip.
\vspace{2mm}

Communication and synchronization between threads can be a performance problem. The system should have efficient means for these purposes, including speculative synchronization.
\vspace{2mm}

It is probably not worthwhile to allow multiple threads to share the same CPU core and level-1 cache simultaneously (this is what Intel calls hyper-threading) because this could allow a low priority thread to steal resources from a high priority thread, and it is difficult for the operating system to determine which threads might be competing for the same execution resources if they are run in the same CPU core.
\vspace{2mm}

\section{Security features} \label{securityFeatures}
Security is included in the fundamental design of both hardware and software. This includes the following features.

\begin{itemize}
\item A flexible and efficient memory protection mechanism.

\item Optional separation of call stack and data stack so that return addresses cannot be compromised by buffer overflow.

\item Each thread has its own protected memory space, except where compatibility with legacy 
software requires a shared memory space for all threads in an application.

\item Device drivers and system functions have carefully controlled access rights. These functions do not have general access to application memory, but only to a specific block of memory that an application may share with a system function when calling it. A device driver has only access to a specific range of input/output ports and system registers as specified in the executable file header and controlled by the system core.

\item A fault in a device driver should not generate a ``blue screen of death'', but generate an error message and close the application that called it and free its resources.

\item Application programs have only access to specific resources as specified in the executable file header and controlled by the system.

\item Array bounds checking is simple and efficient, using an addressing mode with built-in bounds checking or a conditional trap.

\item Various optional methods for checking integer overflow.

\item There is no ``undefined'' behavior. There is always a limited set of permissible responses to an error condition.
\end{itemize}

\subsection{How to improve the security of applications and systems}
Several methods for improving security are listed below. These methods may be useful in ForwardCom applications and operating systems where security is important.

\subsubsection{Protect against buffer overflow}
Input buffers must be protected against overflow. If a software-based protection is not sufficient then you may allocate an isolated block of memory for the input buffer. See page \pageref{isolatedMemoryBlocks}.

\subsubsection{Protect arrays} Array bounds should be checked.

\subsubsection{Protect against integer overflow} Use one of the methods for detecting integer overflow mentioned on page \pageref{integerOverflowDetection}.

\subsubsection{Protect thread memory} Each thread in an application should have its own protected memory space. See page \pageref{threadMemoryProtection}.

\subsubsection{Protect code pointers}
Function pointers and other pointers to code are vulnerable to control flow hijack attacks. These include:

\begin{description}
\item[Return addresses.] Return addresses on the stack are particularly vulnerable to buffer overflow attacks. Use a dual stack design to isolate the return stack from other data.

\item[Jump tables.] Switch/case multiway branches are often implemented as tables of jump addresses. These should use the jump table instruction with the table placed in the CONST section with read-only access. See page \pageref{jumpTableInstruction}.

\item[Virtual function tables.] Programming languages with object polymorphism, such as C++, use tables of pointers to virtual functions. These should use the call table instruction with the table placed in the CONST section with read-only access. See page \pageref{jumpTableInstruction}.

\item[Procedure linkage tables.] Procedure linkage tables, import tables and symbol interposition are not used in ForwardCom. See page \pageref{libraryLinkMethods}.

\item[Callback function pointers.] If a function receives a pointer to a callback function as parameter, then keep this pointer in a register rather than saving it to memory.

\item[State machines.] If a state machine or similar algorithm is implemented with function pointers then place these function pointers in a constant array, use a state variable as index into this array and check the index for overflow. The compiler should have support for defining an array of relative function pointers in the CONST section and access them with the call table instruction.

\item[Other function pointers.] Most uses of function pointers can be covered by the methods described above. Other uses of function pointers should be avoided in high security applications, or the pointers should be placed in protected memory areas or with unpredictable addresses. (See
\href{http://dslab.epfl.ch/proj/cpi/}{Code-Pointer Integrity link}).

\end{description}

\subsubsection{Control access rights of application programs} 
The executable file header of an application program should include information about which kinds of operations the application needs permission to. This may include permission to various network activities, access to particular sensitive files, permission to write executable files and scripts, permission to install drivers, permission to spawn other processes, permission to inter-process communication, etc. The user should have a simple way of checking if these access rights are acceptable. We may implement a system for controlling the access rights of scripts as well. Web page scripts should run in a sandbox.

\subsubsection{Control access rights of device drivers} 
Many operating systems are giving very extensive rights to device drivers. Rather than having a bureaucratic centralized system for approval of device drivers, we should have a more careful control of the access rights of each device driver. The system call instruction in ForwardCom gives a device driver access to only a limited area of application memory (see page \pageref{systemCallInstruction}). The executable file header of a device driver should have information about which ports and system registers the device driver has access to. The user should have a simple way of checking if these access rights are acceptable.

\subsubsection{Standardized installation procedure} 
Malware protection should be an integral part of the operating system, not a third-party add on. The operating system should provide a standardized way of installing and uninstalling applications. The system should refuse to run any program, script or driver that has not been installed through this procedure. This will make it possible for the user to review the access requirements of all installed programs and to remove any malware or other unwanted software through the normal uninstallation procedure.


\end{document}
