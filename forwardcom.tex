\documentclass[11pt,a4paper,oneside,openright]{report}

% compile with XeLatex or LuaLatex, not PDFLatex

\usepackage[bindingoffset=5mm,left=20mm,right=20mm,top=20mm,bottom=20mm,footskip=10mm]{geometry}
\usepackage[utf8x]{inputenc}
\usepackage{hyperref}
\usepackage[english]{babel}
\usepackage{listings}
\usepackage{subfiles}
\usepackage{longtable}
\usepackage{multirow}
\usepackage{ragged2e} 
\usepackage{cmap} % avoid fi ligatures in pdf file
\usepackage{amsthm} % example numbering
\usepackage{color}
\usepackage[T1]{fontenc} % fix problem with underscore not searchable
\usepackage{fontspec}
\defaultfontfeatures{Mapping=tex-text}
%\setmainfont{Verdana}
\setmainfont{Arial}
\setsansfont{Arial}
\renewcommand{\familydefault}{\sfdefault}


% modify style
\newtheorem{example}{Example}[chapter]  % example numbering
\lstset{language=C}                     % formatting for code listing
\lstset{basicstyle=\ttfamily,breaklines=true}
\definecolor{darkGreen}{rgb}{0,0.4,0}
\lstset{commentstyle=\color{darkGreen}}

\begin{document}
\title{ForwardCom: An open-standard instruction set for high-performance microprocessors}
\author{Agner Fog}
\date{\today}
\maketitle
\RaggedRight

\tableofcontents
\setcounter{secnumdepth}{1}



% Introduction
% - Highlights
% - Background
% - Design goals
% - Comparison with other open instruction sets
% - References and links
\subfile{fwc_introduction.tex}

% Basic architecture
% - A fully orthogonal instruction set
% - Instruction size
% - Register set
% - Vector support
% - Vector loops
% - Maximum vector length
% - Instruction masks
% - Addressing modes
\subfile{fwc_basic_architecture.tex}

% Instruction formats
% - Formats and templates
% - Coding of operands
% - Coding of masks
% - Format for jump, call and branch instructions
% - Assignment of opcodes
\subfile{fwc_instruction_formats.tex}

% Instruction lists
% - List of multi-format instructions
% - List of tiny instructions
% - List of single-format instructions
\subfile{fwc_instruction_lists.tex}

% - Description of instructions
% - Common operations that have no dedicated instruction
% - Unused instructions
\subfile{fwc_description_of_instructions.tex}

% Other implementation details
% - Endianness
% - Implementation of call stack
% - Floating point errors and exceptions
% - Detecting integer overflow
% - Multithreading
% - Security features
\subfile{fwc_other_implementation_details.tex}

% Programmable application-specific instructions
\subfile{fwc_programmable_instructions.tex}

% Microarchitecture and pipeline design
\subfile{fwc_microarchitecture_and_pipeline.tex}

% Memory model
% - Thread memory protection
% - Memory management
\subfile{fwc_memory_model.tex}

% System programming
% - Memory map
% - Call stack
% - System calls and system functions
% - Inter-process calls
% - Error message handling
\subfile{fwc_system_programming.tex}

% Support for multiple instruction sets
% - Transitions between ForwardCom and x86-64
% - Transitions between ForwardCom and ARM
% - Transitions between ForwardCom and RISC-V
\subfile{fwc_multiple_instruction_sets.tex}

% Standardization of ABI and software ecosystem
% - Compiler support
% - Binary data representation
% - Further conventions for object-oriented languages
% - Function calling convention
% - Register usage convention
% - Name mangling for function overloading
\subfile{fwc_abi_standard.tex}

% - Binary format for object files and executable files
% - Function libraries and link methods
% - Library function dispatch system
% - Predicting the stack size
% - Exception handling, stack unrolling and debug information
\subfile{fwc_object_file_format.tex}

% - Binary tools
\subfile{fwc_bintools.tex}

% Programming manual
\subfile{fwc_programming_manual.tex}

% Conclusion
\subfile{fwc_conclusion.tex}

% Revision history
\subfile{fwc_revision_history.tex}

% Copyright notice
\subfile{fwc_copyright_notice.tex}

\end{document}
