% chapter included in forwardcom.tex
\documentclass[forwardcom.tex]{subfiles}
\begin{document}
\RaggedRight

\chapter{Revision history}

\subsubsection{Version 1.06, Not published yet.}
\begin{itemize}
\item Added instruction: increment\_jump\_sabove.
\end{itemize}


\subsubsection{Version 1.05, 2017-01-22.}
\begin{itemize}
\item Systematic description of all instructions.
\item Instruction list updated.
\item Added chapter: Support for multiple instruction sets.
\item Added chapter: Software optimization guidelines.
\item Bit manipulation instructions improved.
\item Shift instructions can multiply float by power of 2.
\item Integer division with different rounding modes.
\item Source of option bits for mul\_add, add\_add and compare instructions modified.
\end{itemize}


\subsubsection{Version 1.04, 2016-12-08.}
\begin{itemize}
\item Instruction formats made more consistent. Template E2 modified.
\item Masking principle changed. Fallback value option. r0 and v0 allowed as masks.
\item Compare instruction has additional features.
\item Conditional jumps modified
\item Several other instructions modified.
\end{itemize}


\subsubsection{Version 1.03, 2016-08-01.}
\begin{itemize}
\item Minor changes and additions to manual.
\item Three new instructions added.
\end{itemize}

\subsubsection{Version 1.02, 2016-06-25.}
\begin{itemize}
\item Name changed to ForwardCom.
\item Moved to github.
\item Various security features added.
\item Support for dual stack.
\item Some instruction formats modified, including more formats for jump and call instructions.
\item System call, system return and trap instructions added.
\item New addressing mode for arrays with bounds checking.
\item Several instructions modified or added.
\item Memory management and ABI standards described in more detail.
\item Instruction list in comma separated file instruction\_list.csv.
\item Object file format defined in file elf\_forwardcom.h
\end{itemize}

\subsubsection{Version 1.01, 2016-05-10.}
\begin{itemize}
\item The instruction set is given the name CRISC1.
\item The length of a vector register is stored in the register itself. The basic code structure is modified as a consequence of this. Function calling conventions are also simplified as a consequence of this.
\item All user-level instructions are defined.
\item The entire text has been rewritten and updated.
\end{itemize}

\subsubsection{Version 1.00, 2016-03-22.}
This document is the result of a long discussion on 
\href{http://www.agner.org/optimize/blog/read.php?i=421}{Agner Fog's blog}
, starting on 2015-12-27, as well as input from the RISC-V mailing list and the Opencores forum.
\vspace{2mm}

Additional inspiration was found in various sources listed on page \pageref{referencesToIntroduction}. 
\vspace{2mm}

Version 1.00 of this manual was published at 
\href{http://www.agner.org/optimize}{www.agner.org/optimize}.


\end{document}
